% This must be in the first 5 lines to tell arXiv to use pdfLaTeX, which is strongly recommended.
\pdfoutput=1
% In particular, the hyperref package requires pdfLaTeX in order to break URLs across lines.

\documentclass[11pt]{article}

% Remove the "review" option to generate the final version.
\usepackage[review]{acl}

% Standard package includes
\usepackage{times}
\usepackage{latexsym}

% For proper rendering and hyphenation of words containing Latin characters (including in bib files)
\usepackage[T1]{fontenc}
% For Vietnamese characters
% \usepackage[T5]{fontenc}
% See https://www.latex-project.org/help/documentation/encguide.pdf for other character sets

% This assumes your files are encoded as UTF8
\usepackage[utf8]{inputenc}

% This is not strictly necessary, and may be commented out,
% but it will improve the layout of the manuscript,
% and will typically save some space.
\usepackage{microtype}

% If the title and author information does not fit in the area allocated, uncomment the following
%
%\setlength\titlebox{<dim>}
%
% and set <dim> to something 5cm or larger.

\title{Instructions for Submitting a \\ Project Description to the \\
Archaeology of Intelligent Machines \\
		\hfill \\
		\hfill \\
		\small{1st Semester of 2023-2024}}

  \author{First Author \\
  \texttt{email@s.unibuc.ro} \\\And
  Second Author \\
  \texttt{email@s.unibuc.ro} \\\And
  Third Author \\
  \texttt{email@s.unibuc.ro} \\}

\begin{document}
\maketitle
\begin{abstract}
\textbf{Important to read:} this document is the template that contains the report of your project. The abstract contains a short summary of your theme and report. It is important to submit the description in English and to write at least two pages. If you have difficulties in writing in English, Romanian may also be used. It is mandatory to add an Abstract, Sections \ref{section:intro}, \ref{section:approach}, \ref{section:limitations}, \ref{section:conclusions}, and References.
\end{abstract}


\section*{How to start}
\begin{enumerate}
	\item gather some colleagues and make a team of \textbf{maximum 3 people}
	\item choose a topic that you would like to research (see the project list on the website or propose your own)
	\item make sure your topic does not overlap with other topics that are in progress and that have been chosen by your colleagues
	\item email Sergiu (\url{sergiu.nisioi@unibuc.ro}) to announce your team, your proposal and to discuss how to approach it
	\item after you obtain the approval, start working on it
	\item prepare the project, the report (using this template), and a presentation
	\item place everything in a digital storage space somewhere: a git repo, a google drive, some file on a server etc.; don't send large files by email, send only URLs
	\item current deadline \textbf{January 20}
\end{enumerate}

\section*{Evaluation Criteria}
Project members are evaluated individually. If there's a person who did not have any contribution to the project, they risk failing the course.
\begin{enumerate}
	\item 50\% open-source code to reproduce the results
	\item 25\% a report in English using the current template 
	\item 25\% how you talk about / present your contribution per individual member (each member must say something), contribution to the project is measured per individual member and should be highlighted in the report
\end{enumerate}

\subsubsection*{What is this thing called "contribution"?}
You may create something new, extend something already created, or understand $\rightarrow$ teach $\rightarrow$ present something that already exists \textbf{in relation to topics discussed in class}. 

Use your imagination. Below are a few examples:
\begin{enumerate}
	\item creates something from scratch (code, data, resources, evaluation)
	\item add something new to an existing idea, extend it in some way
	\item implement an existing idea that does not have open-source or reproducible code; make it reproducible, add clear instructions
	\item take an existing idea and make a detailed presentation for other people to understand, analyze its building blocks, present each part in details; think of this as a standalone tutorial
	\item take an existing idea and compile a survey of all the things that have been tried in literature (read all the possible literature and summarize it)
\end{enumerate}


Archaeology of Intelligent Machines is an experimental course, there's no good or bad topic that you can choose, you may go deeply into the underlying mathematical foundations of an algortighm, in its history, or in its contemporary socio-political impact.




\newpage

\section{Introduction}
\label{section:intro}

\textbf{This section is mandatory in your report.} In the introduction make sure to cover the following parts:
\begin{itemize}
	\item what is the problem that we are trying to solve
	\item a bullet point list of contributions that we made, per each member
	\item a summary of the approach
	\item why we chose to approach this project
	\item what other research has been done on this idea
	\item citations with previous work like this \cite{Aho:72} and mention in a paragraph what the previous work is about
	\item if your project's aim is to reproduce an existing paper, you \textbf{must}:
	\subitem - try to understand as much as possible what has been done in the paper
	\subitem - highlight what you \textbf{did not} understand from the approach\footnote{It's okay \textbf{not} to know and to admit it (don't be afraid, you will not be evaluated based on what you did not understand.)}
	\item each author must mention what they learned and what they would want to learn in the future, related to this project
\end{itemize}



\section{Approach}
\label{section:approach}

\textbf{This section is mandatory in your report.}  In this section you must describe the approach you have taken to create this project.
Make sure to explain every step in the process, just like you would explain it to your younger self or someone who does not know anything about the topic.

Cover at least the following aspects:
\begin{enumerate}
 	\item a link to a git /drive repository where your code/data is located
	\item what software tools you have used
	\item how long did the training / processing took?
	\item what kind of machine learning / deep learning tools you used and what architecture
	\item what tricks you tried (gradient clipping, batch normalization), if any
	\item an evaluation report of your method
	\item insert tables, images\footnote{if you take images from the web or from an external source, please cite the source appropriately}, or anything that may convince a reader about the validity of your work
\end{enumerate}

\section{Limitations}
\label{section:limitations}
\textbf{This section is mandatory in your report.}
While we are open to different types of limitations, just mentioning that a set of results have been shown for English only probably does not reflect what we expect. In addition, limitations such as low scalability, the requirement of large GPU resources, or other things that inspire crucial further investigation are welcome. 

\section{Conclusions and Future Work}
\label{section:conclusions}

\textbf{This section is mandatory in your report}, but it is more informal, don't be afraid to be honest, the evaluation will biased positively by your honesty. 
Try to cover some of the following aspects:
\begin{itemize}
	\item now that we did this project, is there anything we could have done different?
	\item is there any way of improving this project?
	\item did we like this project or not? (seriously)
	\item did we learn anything new by doing this?
	\item suggestions for future projects at this course
	\item how could things have been more enjoyable?
\end{itemize}


\newpage


\section{About this template}

The templates include the \LaTeX{} source of this document (\texttt{acl.tex}),
the \LaTeX{} style file used to format it (\texttt{acl.sty}),
an ACL bibliography style (\texttt{acl\_natbib.bst}),
an example bibliography (\texttt{custom.bib}),
and the bibliography for the ACL Anthology (\texttt{anthology.bib}).



\subsection{Compiling \LaTeX{}}

To produce a PDF file, pdf\LaTeX{} is strongly recommended (over original \LaTeX{} plus dvips+ps2pdf or dvipdf). Xe\LaTeX{} also produces PDF files, and is especially suitable for text in non-Latin scripts.

Use any engine, including overleaf to generate a pdf file. Use that pdf file to submit your final project.

\subsection{Preamble}

The first line of the file must be
\begin{quote}
	\begin{verbatim}
	\documentclass[11pt]{article}
	\end{verbatim}
\end{quote}

To load the style file in the review version:
\begin{quote}
	\begin{verbatim}
	\usepackage[review]{acl}
	\end{verbatim}
\end{quote}
For the final version, omit the \verb|review| option:
\begin{quote}
	\begin{verbatim}
	\usepackage{acl}
	\end{verbatim}
\end{quote}

To use Times Roman, put the following in the preamble:
\begin{quote}
	\begin{verbatim}
	\usepackage{times}
	\end{verbatim}
\end{quote}
(Alternatives like txfonts or newtx are also acceptable.)

Please see the \LaTeX{} source of this document for comments on other packages that may be useful.

Set the title and author using \verb|\title| and \verb|\author|. Within the author list, format multiple authors using \verb|\and| and \verb|\And| and \verb|\AND|; please see the \LaTeX{} source for examples.

By default, the box containing the title and author names is set to the minimum of 5 cm. If you need more space, include the following in the preamble:
\begin{quote}
	\begin{verbatim}
	\setlength\titlebox{<dim>}
	\end{verbatim}
\end{quote}
where \verb|<dim>| is replaced with a length. Do not set this length smaller than 5 cm.

\subsection{Document Body}

\subsection{Footnotes}

Footnotes are inserted with the \verb|\footnote| command.\footnote{This is a footnote.}

\subsection{Tables and figures}

See Table~\ref{tab:accents} for an example of a table and its caption.
\textbf{Do not override the default caption sizes.}

\begin{table}
	\centering
	\begin{tabular}{lc}
		\hline
		\textbf{Command} & \textbf{Output}\\
		\hline
		\verb|{\"a}| & {\"a} \\
		\verb|{\^e}| & {\^e} \\
		\verb|{\`i}| & {\`i} \\ 
		\verb|{\.I}| & {\.I} \\ 
		\verb|{\o}| & {\o} \\
		\verb|{\'u}| & {\'u}  \\ 
		\verb|{\aa}| & {\aa}  \\\hline
	\end{tabular}
	\begin{tabular}{lc}
		\hline
		\textbf{Command} & \textbf{Output}\\
		\hline
		\verb|{\c c}| & {\c c} \\ 
		\verb|{\u g}| & {\u g} \\ 
		\verb|{\l}| & {\l} \\ 
		\verb|{\~n}| & {\~n} \\ 
		\verb|{\H o}| & {\H o} \\ 
		\verb|{\v r}| & {\v r} \\ 
		\verb|{\ss}| & {\ss} \\
		\hline
	\end{tabular}
	\caption{Example commands for accented characters, to be used in, \emph{e.g.}, Bib\TeX{} entries.}
	\label{tab:accents}
\end{table}

\subsection{Hyperlinks}

Users of older versions of \LaTeX{} may encounter the following error during compilation: 
\begin{quote}
	\tt\verb|\pdfendlink| ended up in different nesting level than \verb|\pdfstartlink|.
\end{quote}
This happens when pdf\LaTeX{} is used and a citation splits across a page boundary. The best way to fix this is to upgrade \LaTeX{} to 2018-12-01 or later.

\subsection{Citations}

\begin{table*}
	\centering
	\begin{tabular}{lll}
		\hline
		\textbf{Output} & \textbf{natbib command} & \textbf{Old ACL-style command}\\
		\hline
		\citep{Gusfield:97} & \verb|\citep| & \verb|\cite| \\
		\citealp{Gusfield:97} & \verb|\citealp| & no equivalent \\
		\citet{Gusfield:97} & \verb|\citet| & \verb|\newcite| \\
		\citeyearpar{Gusfield:97} & \verb|\citeyearpar| & \verb|\shortcite| \\
		\hline
	\end{tabular}
	\caption{\label{citation-guide}
		Citation commands supported by the style file.
		The style is based on the natbib package and supports all natbib citation commands.
		It also supports commands defined in previous ACL style files for compatibility.
	}
\end{table*}

Table~\ref{citation-guide} shows the syntax supported by the style files.
We encourage you to use the natbib styles.
You can use the command \verb|\citet| (cite in text) to get ``author (year)'' citations, like this citation to a paper by \citet{Gusfield:97}.
You can use the command \verb|\citep| (cite in parentheses) to get ``(author, year)'' citations \citep{Gusfield:97}.
You can use the command \verb|\citealp| (alternative cite without parentheses) to get ``author, year'' citations, which is useful for using citations within parentheses (e.g. \citealp{Gusfield:97}).

\subsection{References}

\nocite{Ando2005,borschinger-johnson-2011-particle,andrew2007scalable,rasooli-tetrault-2015,goodman-etal-2016-noise,harper-2014-learning}

The \LaTeX{} and Bib\TeX{} style files provided roughly follow the American Psychological Association format.
If your own bib file is named \texttt{custom.bib}, then placing the following before any appendices in your \LaTeX{} file will generate the references section for you:
\begin{quote}
	\begin{verbatim}
	\bibliographystyle{acl_natbib}
	\bibliography{custom}
	\end{verbatim}
\end{quote}

You can obtain the complete ACL Anthology as a Bib\TeX{} file from \url{https://aclweb.org/anthology/anthology.bib.gz}.
To include both the Anthology and your own .bib file, use the following instead of the above.
\begin{quote}
	\begin{verbatim}
	\bibliographystyle{acl_natbib}
	\bibliography{anthology,custom}
	\end{verbatim}
\end{quote}

Please see Section~\ref{sec:bibtex} for information on preparing Bib\TeX{} files.

\subsection{Bib\TeX{} Files}
\label{sec:bibtex}

Unicode cannot be used in Bib\TeX{} entries, and some ways of typing special characters can disrupt Bib\TeX's alphabetization. The recommended way of typing special characters is shown in Table~\ref{tab:accents}.

Please ensure that Bib\TeX{} records contain DOIs or URLs when possible, and for all the ACL materials that you reference.
Use the \verb|doi| field for DOIs and the \verb|url| field for URLs.
If a Bib\TeX{} entry has a URL or DOI field, the paper title in the references section will appear as a hyperlink to the paper, using the hyperref \LaTeX{} package.




\section*{Acknowledgements}
No need to add this to your project, but this template is keeping it to give the right credits.

This document has been adapted
by Steven Bethard, Ryan Cotterell and Rui Yan
from the instructions for earlier ACL and NAACL proceedings, including those for 
ACL 2019 by Douwe Kiela and Ivan Vuli\'{c},
NAACL 2019 by Stephanie Lukin and Alla Roskovskaya, 
ACL 2018 by Shay Cohen, Kevin Gimpel, and Wei Lu, 
NAACL 2018 by Margaret Mitchell and Stephanie Lukin,
Bib\TeX{} suggestions for (NA)ACL 2017/2018 from Jason Eisner,
ACL 2017 by Dan Gildea and Min-Yen Kan, 
NAACL 2017 by Margaret Mitchell, 
ACL 2012 by Maggie Li and Michael White, 
ACL 2010 by Jing-Shin Chang and Philipp Koehn, 
ACL 2008 by Johanna D. Moore, Simone Teufel, James Allan, and Sadaoki Furui, 
ACL 2005 by Hwee Tou Ng and Kemal Oflazer, 
ACL 2002 by Eugene Charniak and Dekang Lin, 
and earlier ACL and EACL formats written by several people, including
John Chen, Henry S. Thompson and Donald Walker.
Additional elements were taken from the formatting instructions of the \emph{International Joint Conference on Artificial Intelligence} and the \emph{Conference on Computer Vision and Pattern Recognition}.

% Entries for the entire Anthology, followed by custom entries
\bibliography{anthology,custom}
\bibliographystyle{acl_natbib}


\end{document}